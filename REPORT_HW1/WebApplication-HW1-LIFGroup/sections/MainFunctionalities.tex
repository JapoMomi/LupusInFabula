\section{Main Functionalities}
%What are the main functionalities of the web app? what services does it offer and how it is organized?

\subsection{How a game session works}
To play the game, all the players need to be registered (selecting a username + password and giving the username) to the Lupus Web App. As said in the previous chapter, there are two type of players:
\begin{itemize}
    \item \textbf{MASTER}: The master directs the game, managing the flow of the game and writing down all the events that have happened.
    \item \textbf{OTHERS}: The other players have a specific role (with special effects) and will play the game, trying to win based on the role that they have. 
\end{itemize}
The app permit to the master to efficiently manage the game and to record all the events that have happened while permit to other players to see their role; both can see the current status of the game (the number of the round, the phase and the players that have died and, off course, who is still alive); the masters also can see which role has every player.
\\
A game session in the app works as follow:
\begin{enumerate}
    \item The master creates the game adding all the player using their username. Then chooses the roles for the game and how many instances of each role there will be (N.B. some roles can have only one instance). After that the game can start.
    \item The app will assign to the players (not to the master) a role at random from the ones selected by the master.
    \item The game starts and the app will guide the master through the game suggesting what it's needed to do and permitting the master to write down what happened during each phase (for example, at night phase, the app will "ask" the game-master which is the target of the wolves).
    \item When the game finishes, the master saves the logs and close the session. The app will create the logs of the game.
\end{enumerate}
All the players, including the master, can see the logs of the game from their personal profile. The profile permit also to visualize the logs of all the matches played and also the personal statistics (how many roles have played and how many wins for example).


\subsection{Functionalities and services}
The main functionalities of the Lupus Web App are:
\begin{itemize}
    \item Management of a game for the master
    \item Visualization of the current status of the game for all the players (and game-master)
    \item Random assignment of the roles to the players
    \item Visualization of personal statistics (can also see the statistics of other players)
    \item Visualization of the logs of a game (and also of all game played)
    \item Visualization of the rules of the game (not needed to be logged)
    \item Visualization of the roles and their corresponding effects
    \item Management of friends list (adding or deleting friends) 

\end{itemize}

%The main services of the Lupus Web App are:


\begin{comment}
    

(DA SISTEMARE)
Le funzionalità principali che questa webapp offre sono:\\
(Tutti gli utenti devono essere registrati alla web app per poter giocare)\\
\begin{itemize}
    \item Gestione partita
    \begin{enumerate}
        \item L'assegnazione dei ruoli avviene tramite la web app, quindi il game-master inserisce gli username dei player al interno di una schermata, e quando la partita inizia i gicatori possono vedere dalla loro dispositivo quale ruolo hanno durante quella partita. 
        Mentre il game-master vedrà di ogni utente quale ruolo ha (magari vedendoli di colore diverso, e se sono morti sono grigi)\\
        \item Il master verrà guidato dalla webapp su quello che deve fare, quindi durante la notte l'app chiederea quale target il ruolo X ha scelto (e.g. I lupi A e B hanno deciso di uccidere: "premere sul nome del player che hanno deciso"), al termine della notte la webapp dirà al master cosa è accaduto durante la notte (e.g. chi è morto), segnando i mordi di colore diverso
        \item Durante la votazione può funzionare in due modi (DA DEFINIRE SE USARE ENTRAMBI)\\
        - Chiede di segnare quale giocatore ha deciso di votare chi, dicendo dopo chi è andato al ballottaggio/morto\\
        - Chiede direttamente chi è stato buttato fuori.
        \item Alla fine della partita viene generato un Log di quello che è accaduto
    \end{enumerate}
    
    \item I players hanno la possibilità di avere amici, ossia utenti che verranno suggeriti dalla webapp durante la creazione della partita
    \item Ogni utente potrà vedere le sue statistiche, e.g. quante partite ha giocato, quante ne ha vinte/perse/rateo, quanto ha giocato, ruoli giocati, etc, inoltre avrà accesso alla cronologia delle sue partite e i relativi log 
\end{itemize}
\end{comment}
